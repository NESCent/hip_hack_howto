\documentclass[letterpaper,11pt]{texMemo}

\usepackage[english]{babel}
\date{6 November, 2015}
\title{Concise guide to the NESCent hackathon model}
\author{Arlin Stoltzfus}
\begin{document}

\renewcommand*\contentsname{Concise guide to planning and staging a NESCent hackathon}
\tableofcontents

\newpage
\section{Planning a NESCent hackathon project}
\subsection{Preconditions}
A vision for a hackathon.

\subsection{Roles involved}
Instigators, Sponsors, Organizers.

\subsection{Outputs}
Scoping statement and plans for Publicity, Recruitment, Supportive technology, Pre-event engagement, Meeting logistics, and Follow-up.

\subsection{Process}
Development of a concrete plan is preceded by 2 other stages, as follows: 
\begin{itemize}
\item	{\em Visioning and sponsorship}. Instigators begin with an idea they feel will attract participants and align with the goals of potential sponsors. They use this vision to secure financial support, which may take anywhere from weeks, to many months in the case of a formal grant proposal. 
\item	{\em Leadership team formation}. Armed with an appealing vision and sponsorship, instigators recruit a leadership team of ~5 organizers enthusiastic about the vision, and dedicated to carrying out the work of planning the hackathon and recruiting participants. Recruiting volunteer organizers may take days to many weeks. 
\item	{\em Plan development}. Working with a fixed budget, and guided by the vision, the organizers delineate the scope of the hackathon and create a concrete plan. This will require 5 to 8 meetings (1 hour each), spread out over as many weeks to allow organizers to research options, make arrangements, and respond to changing circumstances (assume 1 hour of work per person per meeting). 
\begin{itemize}
\item	{\em Scoping}. Choosing the scope is a balancing act between advancing the goals of the sponsors and allowing flexibility for members of the target community to make the most of their participation by leveraging their unique interests and skills. Furthermore, the organizers must be invested in the plan, even when the vision has been pre-determined by an agreement of the instigators with the sponsor. This means that the organizers need some flexibility to interpret the vision in a way that stimulates their commitment.  The end result of scoping is a written statement (e.g., a paragraph of written text) that specifies any technological constraints (e.g., a particular language such as R) and programmatic targets (e.g., phylogenetic models).  
\item	{\em Outreach}. The plan for outreach is mainly focused on recruitment. That is, the organizers want to reach out to a particular community whose members have the capacity and the will to contribute to the goals of the project. The organizers also may wish to publicize the event to a larger community of non-participants. The outreach plan should specify the venues (e.g., web sites, email lists) in which the event will be publicized. 
\item	{\em Recruiting}. The recruitment plan must specify a target community, criteria for participation, and diversity goals, consistent with the amount of support that can be provided. The plan may mix direct invitations with a call for applications. The review of applications is the most time-consuming task for organizers: the exact timing and burden of this task should be considered carefully (see the guidelines for recruitment). An optional but important part of the recruitment plan is to consider whether to support remote participation. 
\item	{\em Supportive technology}. There are many choices for supportive technology in regard to code repositories, shared documents, remote participation, and social networking. Committing to a set of preferred technologies, rather than allowing each hackathon team to make its own choices, will make the project more coherent, improving the ability to provide training, monitor progress, and track outcomes. Ideally, participants and facilitators will commit to a limited set of source-code control systems (e.g., GitHub), social networking strategies (e.g., hash tags), and remote participation strategies (e.g., Google+ hangouts). Participants may require training in the preferred set of technologies. 
\item	{\em Pre-event Engagement}. The organizers provide opportunities for discussion mediated by group audio- or videoconferences, email lists, or issue-trackers. This may seem like a futile exercise given that key participants may ignore it.  However, this step is valuable because it provides (1) an opportunity for participants to introduce themselves, get comfortable with the project, and begin building social ties, (2) an early opportunity to discuss specific ideas, and (3) a crucial forum for organizers to assess needs for training, e.g., pre-event engagement may reveal a need for training in a technology that is important for the hackathon but unfamiliar to most participants.  
\item	{\em Event logistics}. Organizers make all the usual arrangements for a multi-person meeting regarding travel and siting. Institutions such as universities often have staff dedicated to helping people organize meetings. The requirements specific to a NESCent hackathon are (1) a room large enough for plenary sessions with 30 people, allowing for open-space pitching (wall-space or easels for up to 10 posters, allowing freedom of movement), (2) break-out spaces for up to 7 teams, either in separate rooms or in a large room with flexibility to rearrange tables and chairs, (3) wireless internet and power outlets. All of this can be done in a single room of 1200 square feet with moveable chairs and tables (ideally, round ones). A common alternative is for teams to seek out separate breakout spaces close to the main room. 
\item	{\em Travel}. The plan for travel must, at a minimum, provide participants with precise instructions for planning their own travel and lodging. For supported travel, the plan must specify how travel arrangements and reimbursements will be made, consistent with the budget and the requirements of the sponsoring organizations.  
\item	{\em Follow-up}. By their nature, hackathons focus on what can be produced during the event itself: follow-up is optional and secondary. However, the organizers may wish to plan for a report to the sponsor, and they may wish to prepare for the case in which hackathon teams produce something that warrants follow-up activities such as a publication or grant proposal.  
\end{itemize}
\end{itemize}

\newpage
\section{Recruiting for a NESCent hackathon}
\subsection{Preconditions}
A team of organizers committed to executing the existing hackathon plan for outreach and recruitment. 
\subsection{Roles involved}
Organizers
\subsection{Outputs}
A roster of prospective participants committed to attending the event. 
\subsection{Process}
\begin{itemize}
\item	{\em Event publicity}. The organizers delegate one or more of their members to disseminate announcements with information on the location, date and theme of the hackathon. Text developed for this may be re-used in recruitment. Advertising the event well in advance, before the call for applications, is helpful in getting it “on the radar” of potential applicants.
\item	{\em Application form}. Guided by their recruitment plan, organizers make an online application form. This requires a design-implement-evaluate-revise workflow that cannot be done at a single meeting. For the NESCent model we recommend that the form request 
\begin{itemize}
\item	{\em Statement of interest} a brief narrative responsive to the hackathon’s scope, indicating what the applicant hopes to achieve by participating. 
\item	{\em Statement of qualification}: a brief narrative responsive to the hackathon’s scope, describing relevant skills and knowledge, and making verifiable references to accomplishments (e.g., a link to a code repo or a paper).
\item	{\em Open-source commitment}: an acknowledgement that all hackathon products are expected to be open source. 
\item	{\em Full attendance commitment}: an acknowledgement that the applicant is available and intends to attend the entire event from start to finish. 
\item	{\em Diversity information}: gender and status as underrepresented minority. 
\item	{\em Location}: the origin of travel to the event (for purposes of budgeting)
\item	{\em Travel support}: support available from other sources for travel, meals, or lodging (for purposes of budgeting).  
\end{itemize}
\item	{\em Open call}. The organizers draft an open call for participation using welcoming language. The open call is disseminated in venues that reach the target community at least 3 weeks before the application deadline, and is subsequently re-issued one or two times.   
\item	{\em Direct recruitment and invited applications}. The organizers may recruit specific individuals to participate, either by reserving them a seat, or by inviting them personally to apply (we prefer the latter). This approach can be used both to ensure that certain technologies or projects are represented at the hackathon, and as part of a strategy to improve diversity. 
\item	{\em Application review}. The organizing team assigns each member a set of applications to review, such that each applicant receives 3 reviews. Organizers read an application, assign an expected contribution score from 1 (participation is unlikely to augment success of the event) to 3 (participation will surely augment success), and make a brief comment explaining the score. This can be done in a spreadsheet (e.g., in a Google spreadsheet linked to the online application form). Applicants are then ranked by their average score. 
\item	{\em Finalizing a roster}. The organizing team reserves an extended block of time (1.5 to 2 hours) to decide on a roster based on the ranking of applicants, the budget, and any other criteria they wish to apply. An obvious starting point is to order the applicants by rank, and designate how much travel support may be offered to each one, working down the list until the travel money runs out (this can be automated using a spreadsheet calculation). The organizers may then adjust this initial roster to improve the composition of the group with respect to skills, aims, or diversity. The organizers may wish to identify additional applicants to be invited if someone declines an invitation. Applicants to be invited are contacted and urged to respond within a week. Finalizing the roster may take 2 weeks. The end result is a list of people definitely committed to attend the event, with a plan to support their expenses.  
\end{itemize}

\newpage
\section{Managing a NESCent hackathon}
\subsection{Preconditions}
A funded budget, a hackathon plan with a scope, and a roster of participants. 
\subsection{Roles involved}
Facilitators, Trainers, Support staff. 
\subsection{Outputs}
Execution of hackathon plan, including travel and event logistics, resulting in tangible outputs.  
\subsection{Process}
\subsubsection*{Travel arrangements} Travel arrangements begin at least 5 weeks before the event. Participants are sent instructions to arrange travel and lodging as soon as possible, following any guidelines to ensure reimbursement. 
\subsubsection*{Pre-event engagement} In the 2 weeks before the event, organizers create opportunities for participants to communicate in a common forum (email list, issue tracker, teleconference). The purpose is to exchange knowledge and ideas, build community, identify training needs, and introduce supportive technologies. 
\subsubsection*{Planning for Day 1 training} The organizers recruit individuals to provide instruction on topics they identify. The focus is on filling gaps in knowledge to ensure success of the hackathon, including scientific and technical knowledge, and the use of supportive technologies chosen for the event. 
\subsubsection*{Day 1 Facilitation} The activities of day 1 must be orchestrated carefully. In a room full of 30 people who do not know each other well, activities that appear simple (e.g., introductions) can go wrong in ways that waste valuable time and drain energy. The success of Day 1 depends both on the months of preparation that brought together a group of people to work on a topic, and on effective facilitation. 
\begin{itemize}
\item	{\em Opening}. Welcome participants, thank sponsors, reiterate the scope and aims, describe the schedule, and introduce facilitators. Allow 10 minutes.
\item	{\em Introductions}. Ask participants to give name, affiliation, and a sentence about the special skills or knowledge they are bringing to the event. Keep introductions short, allowing no more than 10 minutes for 30 people. 
\item	{\em Presentations}. Introduce designated presenters, who provide instruction and respond to questions. Allow from 1 to 3 hours.  
\item	{\em Facilitated open discussion}. The purpose is to identify opportunities and challenges, and discuss promising ideas. This discussion is the first stage in the development of team projects. Discussion ends when the sense of the room is that there are enough technically feasible, in-scope ideas to provide opportunities for everyone. Facilitators begin by reiterating the scope of the hackathon, and then invite comments that pose important challenges, or suggest projects. Facilitators steer the discussion away from implementation details: once it has been determined that an idea is technically feasible, no further discussion of implementation choices is needed. If facilitators sense that key topics remain un-discussed, or that a block of people is going to have trouble fitting in, they should prompt for broader participation to avoid a subsequent failure in team coalescence. Many people are sensitive to criticism of ideas, and are easily silenced or intimidated, even by responses that are not meant to be personal or intimidating. Facilitators may wish to provide guidance on asking rather than judging or criticizing, e.g. rather than responding to an idea with “this won’t work because of X”, ask the question “how do we work around X?”; rather than saying “nobody will use this” ask “what’s the intended user-base for this?”; rather than saying “that isn’t in scope”, ask “how do you align that with the scope?”.  Allow 60 to 120 minutes.  
\item	{\em Break}. Participants intending to make a pitch need a few moments to collect their thoughts. Allow 15 minutes. 
\item	{\em Pitching}. This is the second stage in development of team projects. The goal is to ensure each participant understands each idea. The facilitator invites champions to present a pitch (concept, proposed approach, tangible outcomes) in less than 3 minutes. Participants are instructed to ask only information questions to understand the pitch and how it aligns with the scope: this is not a time for critique or discussion.  Yet, some champions may abandon pitches at this stage. Allow 30 minutes.   
\item	{\em Team coalescence}. This is the third stage in development of team projects. It ends when every participant is part of a team of 3 to 7 people. The facilitator provides initial instructions, and then steps aside. Champions are asked to create a poster and find a position in the room at least 8 feet from any other poster. Participants are instructed that team coalescence is a multi-way negotiation to ensure a productive fit. One person’s special skills may induce a change in the project to take advantage of those skills. Champions are instructed to welcome potential team members, and to be open to new directions. All participants are instructed to join the team that maximizes their contribution to the hackathon (if they can find no way to contribute, they must choose the project that maximizes their chances to learn). This means that champions may choose to abandon a pitch to join another project. Facilitators may intervene to help novice participants, and to interrupt technical discussions that interfere with being welcoming to newcomers. Allow 30 to 75 minutes.  
\item	{\em Team plan, optional report-out and work sessions}. Facilitators instruct teams to break out to separate areas where they will develop and commit to an initial project plan that specifies tangible outcomes and approach. This is the last stage in forming a hackathon team, and it may be done as a continuation of the previous stage.  Allow 20 minutes. If time allows, teams may begin work. Facilitators may choose to end the day by re-convening everyone to hear brief team reports (allow 20 minutes). 
\end{itemize}
\subsubsection*{Days 2 and later} Except on the last day, participants convene in mid-afternoon to hear teams report on accomplishments and challenges. Facilitators limit discussion to practical topics of broad interest, encourage teams with common problems to work together, encourage stalled teams to change tactics, and remind participants to complete tangible outcomes and document their accomplishments before the hackathon ends, because they likely will not work on this when they return to their “day jobs”. 
\subsubsection*{Wrap-up} Participants convene to hear final team reports and discuss issues of broad interest, such as plans for follow-up.  

\newpage
\section{Supporting remote participation (optional)}
\subsection{Preconditions}
A list of individuals committed to participating remotely, and a commitment by organizers to support them. 
\subsection{Roles involved}
Participants, Remote participants, Organizers. 
\subsection{Outputs}
Integration of remote participants into day 1 activities and a team project.  
\subsection{Process}
\subsubsection*{Deciding whether to support remote participation} Even if the hackathon is advertised as a face-to-face event, individuals may contact the organizers to ask if they can participate remotely, due to personal or professional reasons that prevent travel.  The organizers should consider this decision carefully, in advance. Remote participants will not hear or see all of what happens in the meeting, and they will miss out on the shared social environment of extra-meeting activities. Although they will not be able to participate fully in the team-formation process, they can join a team of their choice on day 1. Effective remote participation requires technical support and constant attention from participants. Nevertheless, remote participants may benefit from participation, and hackathon teams may benefit from their contributions. Remote participants often have a specific, justified reason for wanting to participate, e.g., they foresee a role on a likely hackathon team based on ideas that are already being discussed in the community.  
\subsubsection*{Communication strategy} Given that the use of a VTC is almost never feasible, we recommend communication using peer-to-peer technologies for video-, audio-, and text chats. In the avatar strategy, the remote participant, via an extra laptop with a live video connection, virtually sits in the room with the camera facing the plenary speaker or the team. In the alternative buddy-system strategy, the remote participant is paired with an in-person participant who maintains a continuous connection by video chat or other means. We find the buddy system more effective, especially when it leverages a pre-existing relationship between the two partners.  
\subsubsection*{Coordination strategy} Coordination of activities and attention is required for plenary sessions and team-based work sessions. When the buddy system is used, the on-site partner serves as a conduit to relate questions and comments to other participants. To maintain coordination, facilitators keep the meeting on schedule, and notify remote participants of any changes. If Day 1 speakers link their presentation materials to an online agenda, remote participants can follow along.  
\subsubsection*{Managing commitment and expectations} The same reasons that prevent a participant from attending in person may prevent effective participation (e.g., schedule conflicts, deadlines). We recommend asking remote participants to commit to a daily schedule, with a limited allowance for time-zone differences. Facilitators make all participants aware of the importance of supporting remote participants by staying on schedule, and by allowing extra time to communicate with off-site participants.  

\newpage
\section{Hackathon follow-ups}
\subsection{Preconditions}
A hackathon event has taken place. 
\subsection{Roles involved}
Participants, Organizers. 
\subsection{Outputs}
All reimbursements and budgeted arrangements are executed. A report on the hackathon is provided to the sponsor.  
\subsection{Process}

\newpage
\section{Facilities, equipment and supplies}
\addcontentsline{toc}{subsubsection}{Plenary room layout and furnishings}
\subsubsection*{Plenary room layout and furnishings} 
All hackathon activities can be staged in a single room of sufficient size (>1200 square feet) and sufficient sound-deadening properties so that up to 8 teams can work separately in the room. The room must have a projection screen, chairs for everyone, and enough tables (sized for up to 8 people) for all the teams. A sufficient number of power strips – at least one outlet per participant – must be available, with all cables safely secured (e.g. taped to the floor). Notify international participants of the types of outlets and voltages used.

\addcontentsline{toc}{subsubsection}{Breakout space}
\subsubsection*{Breakout space} 
Optionally, breakout space may be provided separate from the plenary room.  To facilitate inter-group interactions as well as hackathon-wide communication, breakout spaces should all be very close (e.g., same building and floor).
\addcontentsline{toc}{subsubsection}{Audiovisual equipment}
\subsubsection*{Audiovisual equipment} 
The plenary room needs a projector and any required video adapters. Hacking spaces need power outlets, power cords, and access to a wireless network with sufficient capacity.  
\addcontentsline{toc}{subsubsection}{Other equipment and supplies}
\subsubsection*{Other equipment and supplies} 
Pitching requires a dozen marking pens and either flipchart easels or 2-foot-by-3-foot adhesive notes to be stuck to walls (note that cloth-covered walls, brick walls, and certain kinds of paneling will not support adhesive notes). 
\addcontentsline{toc}{subsubsection}{Avatar laptops}
\subsubsection*{Avatar laptops} 
Participants bring their own laptops. Avatar laptops may be used as part of a strategy to support remote participants (see Supporting Remote Participation). 

\end{document}